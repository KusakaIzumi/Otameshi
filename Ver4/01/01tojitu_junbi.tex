% 必要な項目ができた場合は適宜サブセクションを追加してください

%\include{begin}
% イベント名を記入する
\section{当日準備}


% 日時と場所を記入する
% 時刻は4桁で記入すること!
\subsection{日時・場所}
\begin{tabular}{p{2zw}rp{38zw}}
  日時 & : & 2019年4月5日(金) 07:00 $\sim$ 10:30\\ %時間を確認してください
  場所 & : & K101, A109,                                        %教室を確認する
\end{tabular}

% 目的を記入する
\subsection{目的}
準備物や自分の役割の最終確認をする.

% イベントのタイムスケジュールを記入する
% 時刻は必ず4桁(00:00)で記入すること!
% 時間の流れは途切れないように記述する!
\subsection{タイムスケジュール}
\begin{longtable}{p{3zw}p{39zw}} %詳細が決まり次第作業

  $\sim$06:00 & \textbf{◎ 各自起床} \\
        & \ \  \textbullet \ \ 遅刻しそうな人は報告LINEに連絡する \\
        & \ \  \textbullet \ \ 互いにモーニングコールをかけ合おう\\\\

  07:00 & \textbf{◎ 集合部屋(A109)に集合} \\
  	& \ \  \textbullet \ \ 朝礼挨拶をする\\
        & \ \  \textbullet \ \ 出席を確認する\\
        & \ \  \textbullet \ \ 遅刻者への連絡\\
        & \ \  \textbullet \ \ 到着した人から自分の荷物にタグを付ける\\
        & \ \  \textbullet \ \ スタッフの参加費を代表者が集める\\\\

  07:10 & \textbf{◎ 1日目読み合わせ確認} \\
  	& \ \  \textbullet \ \ 各自,印刷した企画書最終版で読み合わせを行う\\
        & \ \  \textbullet \ \ 各自が自分の役割を把握し,不明な点を解消する\\
        & \ \  \textbullet \ \ 同じ水準の知識を共有する\\
        & \ \  \textbullet \ \ 物品の確認を行う\\\\

  08:10 & \textbf{◎ 2日目読み合わせ確認} \\
        & \ \  \textbullet \ \ 1日目同様,読み合わせを行う\\
        & \ \  \textbullet \ \ 手の空いているスタッフは物品を再度確認する\\
        & \ \  \textbullet \ \ 確認後全体の記念撮影を部屋内(A109)で行う\\

  10:15 & \textbf{◎ 受け入れ用の部屋(K101:新入生, 先生)で受け入れ準備} \\
      	& \ \  \textbullet \ \ 受付,見回り,先遣隊,後遣隊,救護車がそれぞれに分かれて,準備にとりかかる\\\\

  10:30 & \textbf{◎ 受付準備開始}\\
\end{longtable}

% イベントに必要な役割と人数を記入する
% 担当者は決定次第追記する
% 記入例 ・司会者 2人(名前1,名前2)

% イベントに必要な物品と個数を記入する
% 記入例 ・マジックペン 10本
\subsection{必要物品}
\begin{itemize}
\item カメラ
\item スタッフ荷物用のタグ
\item 企画書
\end{itemize}

\subsection{備考}

%\include{end}

