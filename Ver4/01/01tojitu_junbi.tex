% 必要な項目ができた場合は適宜サブセクションを追加してください

%\include{begin}
% イベント名を記入する
\section{当日準備}


% 日時と場所を記入する
% 時刻は4桁で記入すること!
\subsection{日時・場所}
\begin{tabular}{p{2zw}rp{38zw}}
  日時 & : & 2019年4月5日(金) 07:00 $\sim$ 10:30\\ %時間を確認してください
  場所 & : & K101                                 %教室を確認する
\end{tabular}

% 目的を記入する
\vspace{-5mm}
\subsection{目的}
準備物や自分の役割の最終確認をする

% イベントのタイムスケジュールを記入する
% 時刻は必ず4桁(00:00)で記入すること!
% 時間の流れは途切れないように記述する!
\vspace{-5mm}
\subsection{タイムスケジュール}
\begin{longtable}{p{3zw}p{39zw}} %詳細が決まり次第作業

  $\sim$06:00 & \textbf{◎ 各自起床} \\
        & \ \  \textbullet \ \ 起床したらslackの\#randomに「起きました」と連絡する \\
        & \ \  \textbullet \ \ 連絡がない場合は電話する \\
        & \ \  \textbullet \ \ 互いにモーニングコールをかけ合おう \\\\

  07:00 & \textbf{◎ 集合部屋(K101)に集合} \\
  	    & \ \  \textbullet \ \ 朝礼挨拶をする \\
        & \ \  \textbullet \ \ 出席を確認する \\
        & \ \  \textbullet \ \ 遅刻者への連絡 \\
        & \ \  \textbullet \ \ 到着した人から自分の荷物にタグを付ける \\
        & \ \  \textbullet \ \ スタッフの参加費(500円ピッタリ)を藤沢が集める \\\\

  07:10 & \textbf{◎ 搬入開始} \\
        & \ \  \textbullet \ \ 小島,小松, 三浦, 西森は車を東ロータリーに停めておく \\
        & \ \  \textbullet \ \ K101から物品を車に積み込む \\
        & \ \  \textbullet \ \ 先遣隊と後遣隊は, 2:荷物の搬入へ \vspace{5mm} \\
        
  07:20 & \textbf{◎ 1日目読み合わせ確認} \\
        & \ \  \textbullet \ \ 先遣隊,後遣隊以外のスタッフはK101へ戻って読み合わせを行う \\
  	    & \ \  \textbullet \ \ 各自,企画書最新版を印刷しておく \\
        & \ \  \textbullet \ \ 各自が自分の役割を把握し,不明な点を解消する \\
        & \ \  \textbullet \ \ 同じ水準の知識を共有する \\
        & \ \  \textbullet \ \ 物品の確認を行う \\\\

  08:20 & \textbf{◎ 2日目読み合わせ確認} \\
        & \ \  \textbullet \ \ 1日目同様,読み合わせを行う \vspace{5mm} \\
        
  09:30 & \textbf{◎ その他確認} \\
        & \ \  \textbullet \ \ 読み合わせ後,不安な部分の再確認や司会の練習などを行う \\
        & \ \  \textbullet \ \ 手の空いている人は物品確認を行う \vspace{5mm} \\

  10:15 & \textbf{◎ 受け入れ準備} \\
      	& \ \  \textbullet \ \ 受付,見回り,後遣隊,救護車がそれぞれに分かれて,準備にとりかかる \\\\

  10:30 & \textbf{◎ 受付準備開始} \\
\end{longtable}

% イベントに必要な役割と人数を記入する
% 担当者は決定次第追記する
% 記入例 ・司会者 2人(名前1,名前2)

% イベントに必要な物品と個数を記入する
% 記入例 ・マジックペン 10本
\subsection{必要物品}
\begin{itemize}
\item スタッフ荷物用のタグ:約50個
\item 企画書
\item 室内用シューズ(スリッパ可):1人1足
\end{itemize}

\subsection{備考}
企画書とタイムテーブルを印刷しておく

%\include{end}
