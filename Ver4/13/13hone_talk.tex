% --*- coding:utf-8-unix mode:latex -*--
%\include{Begin}
%%%%%%%%%%%%%%%%%%%%%%%%%%%%%%%%%%%%%%%%%%%%%%%%%%%%%%%%%%%%%%%%%%%%%%%%%%%%%%%

\section{本音トーク}

\subsection{日時・場所}

\begin{tabular}{p{2zw}rp{38zw}}
  日時 & : & 2019年4月5日(金) 21:00 $\sim$ 22:00\\
  場所 & : & くろしお棟 1,2,3の1階
\end{tabular}

\subsection{目的}
新入生からスタッフへ各班ごとのフリートークでは聞けないような話などを本音も交えながら楽しくおしゃべりをする.

\subsection{タイムスケジュール}
% 時刻は必ず4桁(00:00)で書くこと!!!
\begin{longtable}{p{3zw}p{39zw}}
  21:00 & \textbf{◎ フリートーク開始} \\
        & \ \ \textbullet \ \ 各棟のスタッフの進行によりフリートークを開始する\\
        & \ \ \textbullet \ \ 話す内容は司会に任せる(ただし,ギャンブル・飲酒など新入生に悪影響を与えそうな内容は極力避ける)\\\\

  22:00 & \textbf{◎ フリートーク終了} \\
        & \ \ \textbullet \ \ お菓子や飲み物のゴミは分別し,洗えるものは洗い,各くろしお棟の入口付近にまとめて置いておく\\
        & \ \ \textbullet \ \ 翌日の起床時間,起きてからのタイムスケジュールを軽く説明し,寝るよう促す\\
        & \ \ \textbullet \ \ この際見回りにくることや,これ以降の部屋の出入りを禁止することも伝える\\
\end{longtable}

\subsection{人員配置}
\begin{itemize}
 \item くろしお棟1:松尾,小松,角原,徳石,渡辺,日下
 \item くろしお棟2:立岩,平松,松林,斎藤,生野,藤田
 \item くろしお棟3:小谷,明神,藤沢,堀川,長通,宮尾
 \item ヘルプ(呼ばなくてもいい):以西,上村,山口,下出,三浦
\end{itemize}

\subsection{必要物品}
\begin{itemize}
  \item お菓子
  \item 飲み物
  \item ゴミ袋
  \item 新入生に話すネタ
\end{itemize}

\subsection{備考}
\begin{itemize}
  \item 教職員が参加される可能性がある
  \item 場が繋げなくなったら宴にいる先輩(ヘルプ班)に助けを依頼する
\end{itemize}

%%%%%%%%%%%%%%%%%%%%%%%%%%%%%%%%%%%%%%%%%%%%%%%%%%%%%%%%%%%%%%%%%%%%%%%%%%%%%%%

%\include{End}
