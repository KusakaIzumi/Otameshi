%\include{begin}

\section{バス内(帰り)}

\begin{tabular}{p{2zw}rp{38zw}}
  日時 & : & 2019年4月6日(土) 13:30 $\sim$ 16:15\\
  場所 & : & バス内
\end{tabular}

\subsection{目的}
新入生と一緒に二日間を振り返り,共感しあう.

%\subsection{イベント内容}

\subsection{タイムスケジュール}
% 時刻は必ず4桁(00:00)で書くこと!!!
\begin{longtable}{p{3zw}p{39zw}}
  %%14:05 & \textbf{◎ 幡多青少年自然の家出発} \\
  13:30 & \textbf{◎ 幡多少年自然の家出発} \\
        & \ \ \textbullet \ \ 司会者が二日間の感想を話す(野外炊事やイベント,就寝の時の話など)  \\
        & \ \ \textbullet \ \ 新入生に感想などを聞いてみる(新入生が疲れているようなら控えておく) \\
        & \ \ \textbullet \ \ あぐり窪川到着予定時刻とトイレについて話し,後はフリーな時間とする \\\\

  %%14:50 & \textbf{◎ あぐり窪川到着5分前} \\
  14:15 & \textbf{◎ あぐり窪川到着5分前} \\
        & \ \ \textbullet \ \ トイレについての説明と,5分前に集合することを話す(行きと同じ)  \\\\

  %%14:55 & \textbf{◎安芸駅(安芸球場)到着} \\
  14:20 & \textbf{◎ あぐり窪川到着} \\
        & \ \ \textbullet \ \ 休憩時間を伝える\\
        & \ \ \textbullet \ \ 到着の旨を報告slackで連絡する \\\\

  %%15:10 & \textbf{◎ あぐり窪川出発5分前} \\
  14:45 & \textbf{◎ あぐり窪川出発5分前} \\
        & \ \ \textbullet \ \ 司会者が人数チェックする(行きと同様に行う)\\\\


  %%15:15 & \textbf{◎ あぐり窪川出発} \\
  14:50 & \textbf{◎ あぐり窪川出発} \\
	& \ \ \textbullet \ \ 工科大到着時刻と到着後各自解散することを伝える\\
        & \ \ \textbullet \ \ 出発の旨を報告slackで連絡する\\\\

  15:40 & \textbf{◎ 南国道の駅到着5分前} \\
        & \ \ \textbullet \ \ トイレについての説明と,5分前に集合することを話す(行きと同じ)  \\\\

  %%14:55 & \textbf{◎安芸駅(安芸球場)到着} \\
  15:45 & \textbf{◎ 南国の道の駅到着} \\
        & \ \ \textbullet \ \ 休憩時間を伝える\\
        & \ \ \textbullet \ \ 到着の旨を報告slackで連絡する \\\\

  %%15:10 & \textbf{◎ あぐり窪川出発5分前} \\
  15:55 & \textbf{◎ 南国道の駅出発5分前} \\
        & \ \ \textbullet \ \ 司会者が人数チェックする(行きと同様に行う)\\\\


  %%15:15 & \textbf{◎ あぐり窪川出発} \\
  16:00 & \textbf{◎ 南国道の駅出発} \\
	& \ \ \textbullet \ \ 工科大到着時刻と到着後各自解散することを伝える\\
        & \ \ \textbullet \ \ 出発の旨を報告slackで連絡する\\\\

  %%16:00 & \textbf{◎ 工科大到着5分前} \\
  16:10 & \textbf{◎ 工科大到着5分前} \\
      	& \ \ \textbullet \ \ 工科大到着後荷物を持ち,流れ解散であることを伝える \\
        & \ \ \textbullet \ \ 降車の際,名札を回収することを伝える\\
        & \ \ \textbullet \ \ 忘れ物がないように伝える\\
        & \ \ \textbullet \ \ 最後に司会者が締めくくる\\\\

  %%16:05 & \textbf{◎ 工科大到着} \\
  16:15 & \textbf{◎ 工科大到着} \\
        & \ \ \textbullet \ \ 到着の旨を報告slackで連絡する\\
        & \ \ \textbullet \ \ 最初に補助席に座っているスタッフが降車する\\
        & \ \ \textbullet \ \ 降車したスタッフは乗降口で名札を回収する(しおりは各自持ち帰ってもらう)\\
        & \ \ \textbullet \ \ 残りのスタッフは荷物を降ろす\\
        & \ \ \textbullet \ \ 新入生に挨拶をし,見送る(最後の新入生が帰り次第終了する)\\
        & \ \ \textbullet \ \ 新入生が完全に解散した旨を報告slackで連絡する\\
        & \ \ \textbullet \ \ 司会は忘れ物が無いか確認する \\
        & \ \ \textbullet \ \ 記念撮影  \\
\end{longtable}


\subsection{人員配置} %未定
○ 1号車
\begin{itemize}
\item 司会:北村,藤田
\item 1号車受付:伊崎
\item 補助:斎藤,生野,高島,高橋(慎),石野
\end{itemize}

○ 2号車
\begin{itemize}
\item 司会:中島,高橋
\item 2号車受付:吉田
\item 補助:立岩,日下,渡辺,真壁,  別役
\end{itemize}

○ 3号車
\begin{itemize}
\item 司会:丸田,塩谷
\item 3号車受付:江川
\item 補助:青山,新川,角原,小谷,東
\end{itemize}

◯ 救護車
\begin{itemize}
\item 堀川,貞松
\end{itemize}

\subsection{必要物品}
\begin{itemize}
\item 酔い止め薬:各バス1箱
\item エチケット袋:各バス2枚
\item 紙コップ:各バス5個
\item 水(常温):各バス500ml 1本
\item 名札回収用袋:3つ
\end{itemize}


\subsection{備考}


%\include{end}
