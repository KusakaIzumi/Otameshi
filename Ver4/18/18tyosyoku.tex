% --*- coding:utf-8-unix mode:latex -*--

%%%%%%%%%%%%%%%%%%%%%%%%%%%%%%%%%%%%%%%%%%%%%%%%%%%%%%%%%%%%%%%%%%%%%%%%%%%%%%%
%\include{begin}

\section{朝食}

\subsection{日時・場所}

\begin{tabular}{p{2zw}rp{38zw}}
  日時 & : & 2019年4月6日(土) 07:45 $\sim$ 08:30\\
  場所 & : & 食堂・宿泊棟
\end{tabular}


\subsection{タイムスケジュール}
% 時刻は必ず4桁(00:00)で書くこと!!!
\begin{longtable}{p{3zw}p{39zw}}
  07:45 & \textbf{◎ 朝食開始} \\
        & \ \  \textbullet \ \ 準備スタッフ \\
        & \ \ \ \ \ - 朝食をとり次第,宿泊棟に移動し,自分の荷物をまとめる \\
        & \ \ \ \ \ - (???, ???, ???)は荷物の準備ができ次第,自分の荷物を持って第一・二研修室で物品を確保し、体育館に移動する \\
        %& \ \ \ \ \ - 和田は第二集会室,つどいの広間の順で鍵を開ける \\

        & \ \ \ \ \ - 8:15までには体育館に移動し,イベント物品の配置と打ち合わせを行う \\

        & \ \ \textbullet \ \ 誘導を行うスタッフ \\
        & \ \ \ \ \ - 朝のつどいの終了と同時に食堂内で誘導を行うスタッフは急いで食堂に向かって食堂の中の各々のポジションに配置し,奥から詰めて座ることを伝える \\
        & \ \ \ \ \ - ???を先頭に誘導開始する \\
        & \ \ \ \ \ - ???は誘導終了後,お手洗いに行きたい学生がいれば連れて行く \\ %(お手洗いは,浴場横のお手洗いを使用する)
        & \ \ \ \ \ - ???は新入生の列の中間あたりで誘導を行ない,食堂につき次第食堂内の誘導を行う \\
        & \ \ \ \ \ - ???は新入生の一番後ろについて行き,新入生全員が食堂に入ったら(???が新入生をトイレに誘導していないかも確認)朝食をとる \\
        & \ \ \ \ \ - 各スタッフも誘導しながら奥から詰めて座るように指示する \\
        & \ \ \ \ \ - ???は新入生が食堂に全員入ったら報告slackに連絡する \\
        & \ \ \ \ \ - 基本的にスタッフは早めに朝食をとり,補助に回る \\\\

 08:30 & \textbf{◎ 朝食終了(新入生,食堂で仕事のないスタッフ)} \\
        & \ \ \textbullet \ \ 食事を済ませたら各自宿泊棟に戻り,片付けが終わっていなければ片付けの続きを行う \\
\end{longtable}



\subsection{人員配置}
\begin{itemize}
\item 正面広間から食堂までの誘導:下出 高島 小松
\item 食堂内での誘導:高橋(果) 別役 南部
\item イベント準備スタッフ:横田 長道 野田 和田
  \item 先生の誘導(雨天時のみ):高橋(錬) 小野
\end{itemize}



\subsection{備考}
\begin{itemize}
\item イベントスタッフは朝食を早めにとる
\item イベント物品は第二集会室に置いている
\item 美味しく食べる


\end{itemize}

%\include{end}

%%%%%%%%%%%%%%%%%%%%%%%%%%%%%%%%%%%%%%%%%%%%%%%%%%%%%%%%%%%%%%%%%%%%%%%%%%%%%%%
