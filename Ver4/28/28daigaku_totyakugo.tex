% 必要な項目ができた場合は適宜サブセクションを追加してください

%\include{begin}

% イベント名を記入する
\section{大学到着後}


% 日時と場所を記入する
% 時刻は4桁で記入すること!
\subsection{日時・場所}
\begin{tabular}{p{2zw}rp{38zw}}
  日時 & : & 2019年4月6日(土) 16:00 $\sim$ 	\\
  場所 & : & 東ロータリー,K102
\end{tabular}


% イベントのタイムスケジュールを記入する
% 時刻は必ず4桁(00:00)で記入すること!
% 時間の流れは途切れないように記述する!
\subsection{タイムスケジュール}
\begin{longtable}{p{3zw}p{39zw}}
  16:00 & \textbf{◎ 工科大到着} \\\\

  16:15 & \textbf{◎ 新入生見送り終了} \\
        & \ \ \textbullet \ \ 新入生全員の解散の確認後,スタッフの荷物,その他物品をK102に全員で運ぶ \\\\
        & \ \ \textbullet \ \ 日下はゴミ庫の鍵を警備室に取りに行く \\

  16:20 & \textbf{◎ 集合写真撮影} \\
        & \ \ \textbullet \ \ 新入生を見送り次第,スタッフ全員で集合写真を撮る \\
        & \ \ \textbullet \ \ 撮影者は栗原先生にやっていただく \\

  16:25 & \textbf{◎ 教室移動後}  \\
        & \ \ \textbullet \ \ 先遣隊,後遣隊はゴミを指定の場所へ捨てに行く \\
        & \ \ \textbullet \ \ 分別ができていないものは分別する \\
        & \ \ \textbullet \ \ 各研究室,個人,教務部から借りた物品が揃っていることを確認し,順次返却を行う(教務部への返却は藤沢,横田が行う) \\
        & \ \ \textbullet \ \ 返却先が不在等で返却できない場合,一時的に門田研究室にて荷物を保管しておき,後日代表陣が物品の返却を行う \\\\

  16:40 & \textbf{◎ 反省会} \\\\

  17:00 & \textbf{◎ 解散!!} \\
\end{longtable}

\subsection{人員配置}
教務部への物品返却:藤沢,横田

% イベントに必要な物品と個数を記入する
% 記入例 ・マジックペン 10本
\subsection{必要物品}
物品用チェックリスト


% 注意事項やスタッフに周知しておくべきことがあれば記入する
\subsection{備考}
スタッフの皆さん本当にお疲れ様でした.
そして,ありがとうございました.
帰ってゆっくりと体を休めてください.
打ち上げがあるので是非参加してくださいね.
打ち上げで会いましょう!

%\include{end}

