%\include{begin}

\section{後遣隊}

\subsection{日時・場所}

\begin{tabular}{p{2zw}rp{38zw}}
  日時 & : & 2019年4月5日(金) 12:30 $\sim$ 15:00\\
  場所 & : & K101, 吉田研究室,車内
\end{tabular}

\subsection{タイムスケジュール}
% 時刻は必ず4桁(00:00)で書くこと!!!


\begin{longtable}{p{3zw}p{39zw}}
  12:30 & \textbf{◎ 片付け} \\
        & \ \ \textbullet \ \ K101の片付けをする \\
        & \ \ \textbullet \ \ その後,忘れ物の確認をする \\
        & \ \ \textbullet \ \ 会計係藤沢は参加費を吉田研究室に持って行き,保管しておく \\\\

  12:50 & \textbf{◎ 出発準備} \\
        & \ \ \textbullet \ \ 忘れ物を車に運ぶ \\
        & \ \ \textbullet \ \ この時間までに遅刻者の対応を終わらせる \\
        & \ \ \textbullet \ \ 忘れ物がないか最終確認をする \\
        & \ \ \textbullet \ \ K101を退出するとき,電気の消灯などの確認をする \\\\

  13:00 & \textbf{◎ 出発} \\
        & \ \ \textbullet \ \ 西森車は途中で飲み物を冷やすための氷を買う \\
        & \ \ \textbullet \ \ 先遣隊がリストアップした不足物がある場合は購入する \\\\

  16:00 & \textbf{◎ 幡多到着} \\
        & \ \ \textbullet \ \ 遅刻者がいない場合 \\
        & \ \ \ \ \ - 幡多到着後,自分の荷物を第一・二研修室に置く \\
        & \ \ \ \ \ - その後,本隊と合流する \\
        & \ \ \textbullet \ \ 遅刻者がいる場合 \\
        & \ \ \ \ \ - 幡多青少年の家付近になったら,鍵係と連絡をとる \\
        & \ \ \ \ \ - 到着後,鍵係に荷物置きの部屋を開けてもらい,遅刻した新入生の荷物と自分の荷物を置く \\
        & \ \ \ \ \ - 遅刻した新入生を連れて本隊と合流する \\
\end{longtable}


\subsection{人員配置}
\begin{itemize}
\item 後遣隊(西森車):西森,藤沢
\subsection{必要物品}
\begin{itemize}
  \item 夜の荷物(おつまみ,皿,紙コップ,キッチンペーパー,新入生用のお菓子,飲み物,懐中電灯4本)
\end{itemize}
\end{itemize}

%\include{end}
