\documentclass[a4j]{jarticle}
\usepackage{color}
\usepackage[dvipdfmx]{graphicx}
\begin{document}

\section{ペーパードミノ(本イベント)}
\subsection{日時・場所}
日時:2019年4月6日(土)09:30〜10:29\\
\ \ \ 場所:体育館\\
\subsection{目的}
新入生全員がチームをつくり,そのチームの仲間と協力して他のチームと競うことで,より
交流を深めて貰うことを目的とする.班内のスタッフは随時新入生が全員映るような写真を撮る.
\subsection{イベント内容(概要)}
\hspace{-5mm}
◎ \textbf{ルール} \\
紙でドミノを作り,倒した枚数・倒れた距離が長いチームから順位を決める \\
最初に1チームに5枚紙を配布し,作戦を練る時間を5分取る.使用する紙の枚数は30枚で,手で切って枚数を増やしても良い \\
なお制限時間は15分とし,紙の折り方・形は制限しない \\
制限時間になると,作るのをやめ合図で倒し始める(突くのは一回だけ,但し1枚目が倒れるまで突いて良い) \\
スタッフが1チームに1人配置しチームの活動の活性を促す(助言はなし).最後に倒れた紙を全チームで一斉にカウントをする \\
◎ \textbf{得点} \\
\ \ \ ゲームは2回戦行う.ポイントの配分は倒した枚数をそのままポイントに入れ,倒れたドミノの長さの順位ごとに点数を入れる.\\
\ \ \ 順位ごとの点数は,上位から32,30,28,26,24,22,20,18,16,14,12,10,8,6,4,2となる.
なお倒れたドミノの長さをLANケーブルで測って印をつけ,全チームのLANケーブルを並べて比べポイントを配分する.
点数が分かり次第,機材(小島)に点数を伝達しデータとして打ち込みスライドに表示する.

\subsection{タイムスケジュール}
\begin{longtable}{p{3zw}p{39zw}}
09:30 & \textbf{◎ ペーパードミノの準備}\\
      & \ \ \textbullet \ \ 目安2分間で準備する(直前の休み時間の10分間を使っても良い)\\
      & \ \ \textbullet \ \ 休憩の間,各班のスタッフを筆頭にゲームの準備を行う\\
      & \ \ \textbullet \ \ 各チームのスタッフは紙10枚と紙60枚を持っておく(この際10枚は作戦タイム用,60枚は作成タイム用である)\\
      & \ \ \textbullet \ \ 総合司会が所定の位置にチームを配置させる\\

09:32 & \textbf{◎ ペーパードミノの説明}\\
      & \ \ \textbullet \ \ 総合司会からマイクをもらいイベント司会(井崎,青山)がゲームの全般の説明を5分でする\\

09:37 & \textbf{◎ 1回目の作戦タイム}\\
      & \ \ \textbullet \ \ 司会が5分間の作戦タイムをとる\\
      & \ \ \textbullet \ \ この際5枚の紙で練習をしてもらい,司会は実況する\\

09:42 & \textbf{◎ 1回目の作成タイム}\\
      & \ \ \textbullet \ \ 5分間経ったら,司会者がゲームスタートの合図を出す\\
      & \ \ \textbullet \ \ 司会者はストップウォッチで15分を計測する\\
      & \ \ \textbullet \ \ 15分間経ったら司会はゲームを辞めさせ,ドミノから手を離すよう促す\\

09:57 & \textbf{◎ 1回目の計測,結果,片付け}\\
      & \ \ \textbullet \ \ 司会の掛け声でドミノを押し一斉に倒してもらう\\
      & \ \ \textbullet \ \ 各班のスタッフはLANケーブルで長さを測りしるしをつける\\
      & \ \ \textbullet \ \ 計測後に各班のスタッフは,司会の掛け声に合わせて数える\\
      & \ \ \textbullet \ \ 司会は3位〜1位のスコアを口頭で発表する\\
      & \ \ \textbullet \ \ スタッフが作成したドミノとゴミを回収する\\

10:04 & \textbf{◎ 2回目の作戦タイム}\\
      & \ \ \textbullet \ \ 司会が3分間の作戦タイムをとる\\
      & \ \ \textbullet \ \ 1回目と同様に動いてもらう\\

10:07 & \textbf{◎ 2回目の作成タイム}\\
      & \ \ \textbullet \ \ 3分間経ったら,司会がゲームスタートの合図を出す\\
      & \ \ \textbullet \ \ 1回目と同様に動いてもらう\\

10:22 & \textbf{◎ 2回目の計測,結果,片付け}\\
    & \ \ \textbullet \ \ 1回目と同様に計測し順位を発表\\
      & \ \ \textbullet \ \ 司会者は1回目と2回目のスコアを足し高い順から発表する\\
      & \ \ \textbullet \ \ スタッフが作成したドミノとゴミを回収する\\
10:29 & \textbf{◎ 終了し休憩}(15分間)
\end{longtable}

\subsection{必要物品}
\begin{itemize}
\item ワイヤレスマイク:2個
\item スピーカー:2個
\item 机:3つ
\item 椅子:1つ
\item プロジェクター:2個(1個が幡多)
\item スクリーン:2個(1個が幡多)
\item クリアファイル:1個
\item LANケーブル:16本
\item PC
\item 集計用紙:16枚
\item サインペン:16本
\item カメラ
\item 紙:1チーム70枚*16(予備を何枚か)
\item ファイル:1チーム1つ
\end{itemize}
\subsection{備考}


\end{document}
