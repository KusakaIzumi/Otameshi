\documentclass[a4j]{jarticle}
\usepackage{color}
\usepackage[dvipdfmx]{graphicx}
\begin{document}

\section{ペーパードミノ(本イベント)}
\subsection{日時・場所}
日時:2019年4月6日(土)09:30〜\\
\ \ \ 場所:体育館\\
\subsection{目的}
新入生全員がチームをつくり,そのチームの仲間と協力して他のチームと競うことで,より
交流を深めて貰うことを目的とする.班内のスタッフは随時新入生が全員映るような写真を撮る. \vspace{-3mm} \\

\subsection{イベント内容(概要)}

◎ \textbf{ルール} \\
紙でドミノを作り,倒した枚数・倒れた距離が長いチームから順位を決める.\\
最初に1チームに5枚紙を配布し,作戦を練る時間は1回目も2回目も3分とする.本番で使用する紙の枚数は30枚で,手で切って枚数を増やしても良い(ただし限度あり). \\
なお制限時間は1回目は7分とし2回目は10分とする.紙の折り方・形は制限しない. \\
制限時間になると,作るのをやめ合図で倒し始める(突くのは一回だけ,但し作成タイム中に倒れる確認をするために突くのは良い). \\
スタッフはチームの活動の活性を促す(助言はなし).\\\\

◎ \textbf{得点・測定方法} \\
ゲームは2回戦行う.ポイントの配分として倒した枚数はそのままポイントになる.\\
倒れたドミノの長さは順位ごとに上位から32,30,28,26,24,22,20,18,16,14,12,10,8,6,4,2がポイントとなる.\\
なお倒れたドミノの長さをLANケーブルで測って印をつけ,全チームのLANケーブルを並べて比べポイントを配分する.\\
具体的な測定方法としては,作成終了後に各グループの代表者2名(新1年生が1名とグループ内のスタッフ1名)が集計用紙(グループ番号と倒れた枚数の書かれた紙)とLANケーブルを持って前に出てくる.\\
この際,2名の代表者はLANケーブルの両端を持ち他のグループと横に並び,長さを比べ長い順に並んでもらう.\\
並び終えると機材(小島)は昇順に集計用紙を回収し得点を反映する.(なお,司会は場をつなぐために前に出てきた新1年生にインタビュー等を行う) \\

\vspace{-5mm}

\subsection{タイムスケジュール}
\begin{longtable}{p{3zw}p{39zw}}
09:30 & \textbf{◎ ペーパードミノの準備} \\
      & \ \ \textbullet \ \ 目安2分間で準備する(直前の休み時間の10分間を使っても良い) \\
      & \ \ \textbullet \ \ 休憩の間,各班のスタッフを筆頭にゲームの準備を行う \\
      & \ \ \textbullet \ \ 各チームのスタッフは紙10枚と紙60枚を持っておく(この際10枚は作戦タイム用,60枚は作成タイム用である) \\
      & \ \ \textbullet \ \ 総合司会が所定の位置にチームを配置させる \\\\

09:32 & \textbf{◎ ペーパードミノの説明} \\
      & \ \ \textbullet \ \ 総合司会からマイクをもらいイベント司会(伊崎,青山)がゲームの全般の説明を5分で行う\\\\

09:37 & \textbf{◎ 1回目の作戦タイム} \\
      & \ \ \textbullet \ \ 司会が3分間の作戦タイムをとる \\
      & \ \ \textbullet \ \ この際5枚の紙で練習をしてもらい,司会は実況する \\\\

09:40 & \textbf{◎ 1回目の作成タイム} \\
      & \ \ \textbullet \ \ 3分間経ったら,司会者がゲームスタートの合図を出す \\
      & \ \ \textbullet \ \ 開始と同時にスライドで7分を計測する \\
      & \ \ \textbullet \ \ 7分間経ったら司会はゲームを止め,ドミノから手を離すよう促す \\\\

09:47 & \textbf{◎ 1回目の計測,結果,片付け}\\
      & \ \ \textbullet \ \ 司会の掛け声でドミノを押し一斉に倒してもらう\\
      & \ \ \textbullet \ \ 各班のスタッフはLANケーブルで長さを測り印をつける\\
      & \ \ \textbullet \ \ 印をつけた後に各グループのスタッフは,ドミノの枚数を数え集計用紙に記入する\\
      & \ \ \textbullet \ \ 印のついたLANケーブルと集計用紙を持った代表者2名(新1年生が1名とグループ内のスタッフ1名)が前に出る\\
      & \ \ \textbullet \ \ 代表者2名の持つLANケーブルの長さを比べ,長い順に並んでもらう\\
      & \ \ \textbullet \ \ 機材(小島)は昇順に集計用紙を回収し得点を反映する\\
      & \ \ \textbullet \ \ スタッフが作成したドミノとゴミを回収する\\

09:57 & \textbf{◎ 2回目の作戦タイム} \\
      & \ \ \textbullet \ \ 司会が3分間の作戦タイムをとる \\
      & \ \ \textbullet \ \ 1回目と同様に動いてもらう \\\\

10:00 & \textbf{◎ 2回目の作成タイム} \\
      & \ \ \textbullet \ \ 3分間経ったら,司会がゲームスタートの合図を出す \\
      & \ \ \textbullet \ \ 1回目と同様に動いてもらう(ただし2回目は10分間なので注意) \\\\

10:10 & \textbf{◎ 2回目の計測,結果,片付け} \\
      & \ \ \textbullet \ \ 1回目と同様に動いてもらう \\
      & \ \ \textbullet \ \ 機材(小島)が1回目と2回目の得点を反映させたスライドを作成中に,司会は場をつなぐ\\
      & \ \ \textbullet \ \ 順位の反映されたスライドを発表\\
      & \ \ \textbullet \ \ スタッフが作成したドミノとゴミを回収する\\
      & \textbf{◎ 終了し休憩(終わり次第10:45分まで休憩)}\\\\
\end{longtable}

\newpage

\subsection{必要物品}
\begin{itemize}
\item ワイヤレスマイク:2個
\item スピーカー:2個
\item 机:3つ
\item 椅子:1つ
\item プロジェクター:2個(1個が幡多)
\item スクリーン:2個(1個が幡多)
\item クリアファイル:1個
\item LANケーブル:16本
\item PC
\item 集計用紙:16枚*2(2回分の作成タイム用)
\item サインペン:16本
\item カメラ
\item 紙:1チーム70枚*16(予備を何枚か)
\item ファイル:1チーム1つ
\end{itemize}
\subsection{備考}


\end{document}
