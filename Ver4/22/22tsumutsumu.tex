\section{ペーパードミノ(本イベント)}
\subsection{日時・場所}
日時:2019年4月6日(日)09:30〜10:29\\
\ \ \ 場所:体育館\\
\subsection{目的}
新入生全員がチームをつくり,そのチームの仲間と協力して他のチームと競うことで,より
交流を深めて貰うことを目的とする.班内のスタッフは随時新入生が全員映るような写真を撮る.
\subsection{イベント内容(概要)}
紙でドミノを作り,倒した枚数・倒れた距離が長いチームから順位を決める.最初に1チームに5枚紙を配布し,作戦を練る時間を5分取る.使用する紙の枚数は30枚で,手で切って枚数を増やしても良い.なお制限時間は15分とし,紙の折り方・形は制限しない.
制限時間になると,作るのをやめ合図で倒し始める(突くのは一回だけ,但し1枚目が倒れるまで突いて良い).スタッフが1チームに1人配置しチームの活動の活性を促す(助言はなし).最後に倒れた紙を全チームで一斉にカウントをする.\\
\ \ \ ゲームは2回戦行う.なお倒れたドミノの長さをビニール紐で測って切り,全チームのビニール紐を並べて比べポイントを配分する.\\
\ \ \ ポイントの配分は倒した枚数をそのままポイントに入れ,倒れたドミノの長さの順位ごとに点数を入れる.順位ごとの点数は,上位から32,30,28,26,24,
22,20,18,16,14,12,10,8,6,4,2となる.点数が分かり次第,イベント司会に点数を伝達しデータとして打ち込みスライドに表示する.

\subsection{タイムスケジュール}
09:30 \ \ \ ◎ \textbf{ペーパードミノの準備}\\
\hspace{15mm}・目安2分間で準備する.\\
\hspace{15mm}・休憩の間,各班のスタッフを筆頭にゲームの準備を行う.\\
\hspace{15mm}・各チームのスタッフは紙10枚と紙60枚を持っておく.\verb+(+この際10枚は作戦タイム用,60\\
\hspace{15mm}枚は作成タイム用である.\verb+)+\\
\hspace{15mm}・総合司会が所定の位置にチームを配置させる.\\
\ \ \ 09:32 \ \ \ ◎ \textbf{ペーパードミノの説明}\\
\hspace{15mm}・5分でゲームの説明を行う.\\
\hspace{15mm}・総合司会がイベント司会に司会権を振り,イベント司会がゲームの全般の説明をする.\\
\ \ \ 09:37 \ \ \ ◎ \textbf{1回目の作戦タイム}\\
\hspace{15mm}・司会が5分間の作戦タイムをとる.\\
\hspace{15mm}・この際5枚の紙で練習をしてもらい,司会は実況する.\\
\ \ \ 09:42 \ \ \ ◎ \textbf{1回目の作成タイム}\\
\hspace{15mm}・5分間経ったら,司会者がゲームスタートの合図を出す.\\
\hspace{15mm}・司会者はストップウォッチで15分を計測する.\\
\hspace{15mm}・15分間経ったら司会はゲームを辞めさせ,ドミノから手を離すよう促す.\\
\ \ \ 09:57 \ \ \ ◎ \textbf{1回目の計測,結果,片付け}\\
\hspace{15mm}・司会の掛け声でドミノを押し一斉に倒してもらう.\\
\hspace{15mm}・各班のスタッフは倒れたドミノの長さを計り,司会の掛け声に合わせて数える.\\
\hspace{15mm}・司会者は3位〜1位のスコアを口頭で発表する.\\
\hspace{15mm}・スタッフが作成したドミノとゴミを回収する.\\
\ \ \ 10:04 \ \ \ ◎ \textbf{2回目の作戦タイム}\\
\hspace{15mm}・司会が3分間の作戦タイムをとる.\\
\hspace{15mm}・1回目と同様に動いてもらう.\\
\ \ \ 10:07 \ \ \ ◎ \textbf{2回目の作成タイム}\\
\hspace{15mm}・3分間経ったら,司会者がゲームスタートの合図を出す.\\
\hspace{15mm}・1回目と同様に動いてもらう.\\
\ \ \ 10:22 \ \ \ ◎ \textbf{2回目の計測,結果,片付け}\\
\hspace{15mm}・1回目と同様に計測し順位を発表.\\
\hspace{15mm}・司会者は1回目と2回目のスコアを足し高い順から発表する.\\
\hspace{15mm}・スタッフが作成したドミノとゴミを回収する.\\
\ \ \ 10:29 \ \ \ ◎ \textbf{終了し休憩}\verb+(+15分間\verb+)+

\subsection{必要物品}
紙:1チーム70枚(予備を何枚か)\\
\ \ \ ファイル:1チーム1つ\\
\ \ \ ビニール紐:長さを測るために1玉
