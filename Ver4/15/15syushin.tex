% 必要な項目ができた場合は適宜サブセクションを追加してください

%\include{begin}

% イベント名を記入する
\section{就寝}

% 日時と場所を記入する
% 時刻は4桁で記入すること!
\subsection{日時・場所}
\begin{tabular}{p{2zw}rp{38zw}}
  日時 & : & 2019年4月5日(金) \\%22:00 $\sim$ 23:40\\
  場所 & : & くろしお棟1,2,3
\end{tabular}


% イベントの概要やルールを記入する

% イベントのタイムスケジュールを記入する
% 時刻は必ず4桁(00:00)で記入すること!
% 時間の流れは途切れないように記述する!
\subsection{タイムスケジュール}
\begin{longtable}{p{3zw}p{39zw}}
  22:00 & \textbf{◎ 消灯呼びかけ(1回目見回り)} \\
        & \ \  \textbullet \ \ 女子:\\
        & \ \  \textbullet \ \ 男子:\\\\

  22:30 & \textbf{◎ 2回目見回り開始} \\
        & \ \  \textbullet \ \ 女子:\\
        & \ \  \textbullet \ \ 男子:\\\\

  23:00 & \textbf{◎ 3回目見回り開始} \\
        & \ \  \textbullet \ \ 女子:\\
        & \ \  \textbullet \ \ 男子:\\\\

  23:30 & \textbf{◎ 4回目見回り開始} \\
        & \ \  \textbullet \ 女子:\\
        & \ \  \textbullet \ \ 男子:\\\\

  23:40 & \textbf{◎ 4回目見回り終了、スタッフ就寝} \\

\end{longtable}


% イベントに必要な役割と人数を記入する
% 担当者は決定次第追記する
% 記入例 ・司会者 2人(名前1、名前2)


% イベントを実施するときに新入生や先生、スタッフがどこに配置するかを記入する
% 図があるとわかりやすい


% イベントに必要な物品と個数を記入する
% 記入例 ・マジックペン 10本

% 目的を記入する
\subsection{人員配置}
\begin{itemize}
\item 女子見張り:
\item 男子見張り:
\end{itemize}


\subsection{必要物品}
\begin{itemize}
\item 懐中電灯:4本
\end{itemize}


\subsection{注意事項}
\begin{itemize}
\item 自動販売機の使用は原則禁止する
%\item 万が一自動販売機へ行きたい新入生がいたら,自動販売機(食堂前とくろしお棟洗濯室の隣)へ誘導しその学生が買い終わるまで見張る(そのまま脱走する可能性を防ぐため)
\item 女子部屋に男子が,男子部屋に女子が行かないように見張る
\item それと同時に男性の先生方が女子部屋に,女性の先生方が男子部屋に行かないように注意する
\item その他何が起こるかわからないため,見回り時に何かあったら報告slackに連絡する
\end{itemize}

% 注意事項やスタッフに周知しておくべきことがあれば記入する
\subsection{備考}
\begin{itemize}
\item 状況に応じて見回りの回数を増やす
\item 車出し担当スタッフは優先的に就寝できるように配慮する
\item 余裕があれば宴に参加する(B3以上)
\item 当日は懐中電灯を第4研修室に置いておくため,22:00までに自分自身で取りに行き,次の日の朝戻す
\item 何かあったら報告slackで報告する
\end{itemize}

%\include{end}
