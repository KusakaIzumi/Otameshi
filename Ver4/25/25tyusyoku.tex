% --*- coding:utf-8-unix mode:latex -*--
%\include{begin}
%%%%%%%%%%%%%%%%%%%%%%%%%%%%%%%%%%%%%%%%%%%%%%%%%%%%%%%%%%%%%%%%%%%%%%%%%%%%%%%

\section{昼食}

\subsection{日時・場所}
\begin{tabular}{p{2zw}rp{38zw}}
  日時 & : & 2019年4月6日(土) 11:55 $\sim$ 13:15 \\
  場所 & : & 食堂
\end{tabular}


\subsection{タイムスケジュール}
% 時刻は必ず4桁(00:00)で書くこと!!!
\begin{longtable}{p{3zw}p{39zw}}
  11:55 & \textbf{◎ 食堂への移動・体育館の片付け} \\
        & \ \ \textbullet \ \ 代表(宮尾)が昼食を食堂で食べることを伝え、スタッフは新入生を連れて,4,5班ずつ食堂へと移動する \\
        %& \ \ \textbullet \ \ 各班のスタッフは弁当配布係(以西,宮尾,江川,真壁,渡辺)とお茶配布係(別役,三浦,日下,渡辺, 高橋(果))から弁当を受け取り,新入生・先生に配布する\\
        %& \ \ \textbullet \ \ 配布次第各班で昼食をとる \\
        & \ \ \textbullet \ \ 先遣隊と後遣隊は,先に体育館の片付けをし,終わり次第食堂で昼食を食べる \\
        & \ \ \textbullet \ \ 食堂内誘導係(角原,小谷)は表彰式が終わったら,素早く食堂へ向かう \\
        & \ \ \textbullet \ \ 手の空いているスタッフは,食堂に着いたら奥から詰めて座ることを伝える \\
        & \ \ \textbullet \ \ 最後に新入生を誘導してきたスタッフは食堂に着いたことを食堂内誘導係の人に伝える \\
        
  12:00 & \ \ \textbullet \ \ 食堂で昼食をとる \\
        & \ \ \textbullet \ \ 食べ終わった人は食堂で待機する \\
        & \ \ \textbullet \ \ バス司会者(塩谷, 中島, 丸田, 高橋, 北村, 藤田)は早めに昼食を済ませ,バス周辺で待機しておく \\
        & \ \ \textbullet \ \ トイレ係(日下,斎藤)はなるべく出入り口に近い位置で昼食をとる \\
        & \ \ \textbullet \ \ トイレ係は食堂から出入りする新入生を見張る \\

  12:30 & \textbf{◎ バス到着予定} \\
        & \ \ \textbullet \ \ バス(運転手:???,???, ???)が到着するのでバス司会は周辺で待機する \\\\

  13:10 & \textbf{◎ バス出発準備アナウンス} \\
        & \ \ \textbullet \ \ 食堂内案内係(角原,小谷)は食器を片付け,トイレを済ませておくようにアナウンスする \\
        & \ \ \textbullet \ \ トイレ係(日下,斎藤)はトイレに行きたくなった新入生を見張る \\
        & \ \ \textbullet \ \ 東は,たばこ吸いに行っている先生方に出発の時間が近づいていることを知らせる \\
        & \ \ \textbullet \ \ 藤田からバス到着の連絡が来たら,小谷は事務室に行って鍵を開けてもらい,
        							1号車から順に第一・二研修室へ荷物を取りに行くことを促し,荷物を取り終わったらバスへ向かってもらうように誘導案内する\\
        & \ \ \textbullet \ \ スタッフは新入生の誘導の補助をしながら第一・二研修室へ荷物を取りに行き,バスへ向かう \\
        & \ \ \textbullet \ \ 角原は全員が食堂を出たら,食堂内に忘れ物がないかを確認し,バスへ向かう \\
        & \ \ \textbullet \ \ バス司会者(塩谷, 中島, 丸田, 高橋, 北村, 藤田)は一人がバスに乗り込む新入生のチェックをする \\
        & \ \ \textbullet \ \ もう一方のバス司会者(塩谷, 中島, 丸田, 高橋, 北村, 藤田)と各号車に乗るスタッフは,新入生と先生の荷物の積みこむ \\
  \end{longtable}


\subsection{人員配置}
\begin{itemize}
\item 食堂内案内係:角原,小谷
\item 体育館の片付け:先遣隊,後遣隊
\item トイレを見張る係:日下,斎藤
\item 新入生・先生誘導係:各班のイベントスタッフ
\item タバコを吸いに行っている先生方へ出発の知らせをする係:東
\item バスへの誘導係:各班スタッフ
\item バス到着連絡係:藤田
\end{itemize}


\subsection{必要物品}
乗車確認リスト:3部


\subsection{備考}
\begin{itemize}
  %\item 各班食べ終わった弁当はまとめておく、班全員が食べ終わったら配布された場所に持っていく
  \item トイレに行きたい人は体育館のトイレに行くように誘導する
  \item 食事はついた人から奥に座って食べる
  \item トイレに行きたい人は食堂近くのトイレに行くようにスタッフが誘導する
  %\item は第一集会室から全員の荷物がなくなったら,第一集会室の鍵を閉め,持っている鍵を室戸事務室に返却する
\end{itemize}




%%%%%%%%%%%%%%%%%%%%%%%%%%%%%%%%%%%%%%%%%%%%%%%%%%%%%%%%%%%%%%%%%%%%%%%%%%%%%%%
%\include{end}
