% 必要な項目ができた場合は適宜サブセクションを追加してください
%\include{begin}

% イベント名を記入する
\section{幡多到着}

% 日時と場所を記入する
% 時刻は4桁で記入すること!
\subsection{日時・場所}
\begin{tabular}{p{2zw}rp{38zw}}
  日時 & : & 2019年4月5日(金) 14:40 $\sim$ 15:25 \\
  場所 & : & バス内 〜 大研修室
\end{tabular}

% イベントのタイムスケジュールを記入する
% 時刻は必ず4桁(00:00)で記入すること!
% 時間の流れは途切れないように記述する!
\subsection{タイムスケジュール(晴天)}
\begin{longtable}{p{3zw}p{39zw}}
  15:10 & \textbf{◎ バス到着・案内} \\
        & \ \   \textbullet \ \ 先遣隊(小松,小島,横田,野田,以西,三浦)は降車場所〜大研修室の道中に誘導員として配置し,
        						新入生の誘導をバス司会スタッフとともに行う(配置場所は図\ref{fig:hare}参照) \\
        & \ \   \textbullet \ \ 本隊はバス到着後, 乗っている新入生の荷物を降ろす \\
        & \ \   \textbullet \ \ 1号車荷卸スタッフ(伊崎,斎藤,高橋(慎))と2号車荷卸スタッフ(吉田,新田,別役)と3号車荷卸スタッフ(小谷,青山,東)はそれぞれ1号車,2号車,3号車の荷卸をする\\
        & \ \   \textbullet \ \ 荷卸の優先順位は高い順からスタッフ→教職員→新入生となる(尚,教職員の荷物はなるべく手渡しとする)\\
        & \ \   \textbullet \ \ 1号車の荷卸スタッフ(伊崎,斎藤,高橋(慎))は1号車の荷卸が高橋(慎)で行えると判断した場合,伊崎,斎藤は速やかに自分の荷物を持ち第一・二研修室に運ぶ \\
        & \ \   \textbullet \ \ 2号車の荷卸スタッフ(吉田,新田,別役)は2号車の荷卸が別役で行えると判断した場合,吉田,新田は速やかに自分の荷物を持ち第一・二研修室に運ぶ \\
        & \ \   \textbullet \ \ 3号車の荷卸スタッフ(小谷,  青山,東)は3号車の荷卸が東で行えると判断した場合,小谷,青山は速やかに自分の荷物を持ち第一・二研修室に運ぶ \\
        & \ \   \textbullet \ \ バスを降りるときにゴミを回収する \\
        & \ \   \textbullet \ \ バスから降りたスタッフは速やかに自分の荷物を第一・二研修室に運ぶ \\
        & \ \   \textbullet \ \ スタッフが下りたら,小野,藤田(M2)は先生を誘導する \\
        & \ \   \textbullet \ \ プラカードを持つスタッフはそのまま大研修室へ,それ以外のスタッフは新入生を誘導する \\
        & \ \   \textbullet \ \ 新入生は到着順に降車させる \\
        & \ \   \textbullet \ \ 荷卸などで持って行くことができないスタッフは各バスの司会に運んでもらう \\
        & \ \   \textbullet \ \ 第一・二研修室配置のスタッフ(以西,小松)は新入生・教職員の荷物を班ごとにまとめて置かせ,大研修室に行くように誘導する(その際しおりは持って行くようにさせる) \\
        & \ \   \textbullet \ \ トイレに行きたい新入生がいる場合,横田,野田はトイレの場所を伝え、行くように促す \\
        & \ \   \textbullet \ \ 各班代表スタッフは迅速に大研修室へ移動する \\
        & \ \   \textbullet \ \ 各班代表スタッフは班名の書かれたプラカードを持ち,教職員と対面させるように新入生の整列を行う \\
        & \ \   \textbullet \ \ 整列後,新入生を先生方のイスに向かせて座らせ,スタッフはステージ向かって右側に移動する \\
        & \ \   \textbullet \ \ 各班代表スタッフは班員が揃い次第司会に報告する \\
        & \ \   \textbullet \ \ 立岩(司会)はどの班が揃っているか適宜確認を行う \\
        & \ \   \textbullet \ \ 全員の荷物が入ったら,以西,小松は事務室へ行き第一・二・四研修室の鍵を閉めてもらう \\\\

\newpage
        
        & \textbf{◎ 救急箱の受け渡し(堀川)}\\
        & \ \ \textbullet \ \ 堀川車から救急箱を降ろしておき,野外炊事が始まる前に塩谷に渡す. \\\\

  15:25 & \textbf{◎ 各班整列完了・教員誘導完了} \\
        & \ \   \textbullet \ \ 各班代表者は大研修室に入ってきた新入生を班ごとに誘導する \\
        & \ \   \textbullet \ \ 出席する新入生・教職員が揃い次第,入所式を開始 \\
        & \ \   \textbullet \ \ 施設職員呼び出し係(東)は、準備が完了しそうになったら施設職員さんを呼びに行く \\
\end{longtable}

\subsection{タイムスケジュール(雨天)}
\begin{longtable}{p{3zw}p{39zw}}
  15:10 & \textbf{◎ バス到着・案内} \\
        & \ \   \textbullet \ \ 先遣隊(小松,小島,横田,野田,以西,三浦)は降車場所〜大研修室の道中に誘導員として配置し,新入生の誘導を行う(配置場所は図\ref{fig:hare}参照) \\
        & \ \   \textbullet \ \ 本隊はバス到着後スタッフ全員でトランクの荷物を玄関に降ろす.(この時,各バスに乗っている新入生と教職員はバス内で待機する) \\
        & \ \   \textbullet \ \ 1号車・2号車・3号車の司会スタッフは荷物を運び終えたらバスで待機している新入生と教職員を誘導しに行く \\
        & \ \   \textbullet \ \ 第一・二研修室配置のスタッフ(以西,小松)は新入生・教職員の荷物を班ごとにまとめて置かせ,大研修室に行くように誘導する(その際しおりは持って行くようにさせる)\\
        & \ \   \textbullet \ \ トイレに行きたい新入生がいる場合, 横田,野田はトイレの場所を伝え、行くように促す \\
        & \ \   \textbullet \ \ 研修生入り口の傘立てに傘を入れる。 \\
        & \ \   \textbullet \ \ 各班代表スタッフは迅速に大研修室へ移動する \\
        & \ \   \textbullet \ \ 各班代表スタッフは班名の書かれたプラカードを持ち,教職員と対面させるように新入生の整列を行う \\
        & \ \   \textbullet \ \ 整列後,新入生を出入り口側に向かせて座らせ,スタッフは右側に移動する \\
        & \ \   \textbullet \ \ 立岩(司会)はどの班が揃っているか適宜確認を行う \\
        & \ \   \textbullet \ \ 全員の荷物が入ったら,以西,小松は事務室へ行き第一・二・四研修室の鍵を閉めてもらう \\\\

  15:25 & \textbf{◎ 各班整列完了・教員誘導完了} \\
        & \ \   \textbullet \ \ 各班代表者は大研修室に入ってきた新入生を班ごとに誘導する \\
        & \ \   \textbullet \ \ 新入生を座らせ司会者へ報告する \\
        & \ \   \textbullet \ \ 出席する新入生・教職員が揃い次第,入所式を開始 \\
        & \ \   \textbullet \ \ 準備が終わりそうになったら,施設職員呼び出し係(東)が幡多職員さんを呼びに行く \\
\end{longtable}

\newpage

% イベントに必要な役割と人数を記入する
% 担当者は決定次第追記する
% 記入例 ・司会者 2人(名前1、名前2)
\subsection{人員配置(人数により調整あり)}
\begin{itemize}

\item 1号車荷卸スタッフ:伊崎,斎藤,高橋(慎)
%山口 三浦 早瀬
\item 2号車荷卸スタッフ:吉田,新田,別役

\item 3号車荷卸スタッフ:小谷,青山,東
%横井 西井 森
%\item ワゴン:生野,別役

\item 教職員誘導:小野,藤田(M2)

\item バス司会:北村,中島,高橋(龍),丸田,塩谷,藤田(B3)
%川添 東 藤田
%\item 袋配布:藤田,日下

\item 施設職員呼び出し係:東
  %島田
\item 司会:立岩
\end{itemize}


% イベントに必要な物品と個数を記入する
% 記入例 ・マジックペン 10本
\subsection{必要物品}
%\begin{itemize}
%\item 靴を入れる袋:200枚
%\item 傘を入れる袋:200枚
%\end{itemize}


% 注意事項やスタッフに周知しておくべきことがあれば記入する

\subsection{備考}
\begin{itemize}
\item 大研修室での整列は,野外炊事の班で固まって集まるようにし,集まったことが確認できたらその場で座りやすいように列をある程度崩してもよい
\item トランクに荷物を入れている荷卸スタッフの荷物は各バス司会が運び第一・二研修室まで運ぶ
\item タグに野外炊事班を書いておく
%\item 晴天時と雨天時でタイムスケジュールが多少異なるので注意する
\item 物品は第四研修室に運ぶ
\item 大研修室はせまいので,できるだけ詰めて座るように促す
\end{itemize}


%\include{end}
