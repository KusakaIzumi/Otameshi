\documentclass[12pt]{jarticle}
\usepackage{docmute}
\usepackage{caption}
\usepackage[subrefformat=parens]{subcaption}
\usepackage[dvipdfmx]{graphicx}
\usepackage{longtable} % 長い表生成用
\usepackage{texilikecover} % 表紙のデザイン用
\usepackage{listliketab}
\usepackage{ascmac}
\usepackage{comment}
\usepackage{here}
\usepackage{mediabb}


%ここから上は触らない
%必要なパッケージを追加する
%\usepackage{docmute}
%\usepackage{caption}
%\usepackage[subrefformat=parens]{subcaption}
%\usepackage[dvipdfmx]{graphicx}
%\usepackage{longtable} % 長い表生成用
%\usepackage{texilikecover} % 表紙のデザイン用
%\usepackage{listliketab}
%\usepackage{ascmac}
%\usepackage{comment}
%\usepackage{here}
%\usepackage{mediabb}

%以上は追加済み

%\usepackage{}


\begin{document}
% 必要な項目ができた場合は適宜サブセクションを追加してください

% イベント名を記入する
%例:\section{大学到着後}
\section{sample}

% 日時と場所を記入する
% 時刻は4桁で記入すること!
\subsection{日時・場所}
\begin{tabular}{p{2zw}rp{38zw}}
	日時 & : & 2016年4月17日(日) 16:00 $\sim$\\ 
	場所 & : & 東ロータリー,K102
\end{tabular}


% イベントのタイムスケジュールを記入する
% 時刻は必ず4桁(00:00)で記入すること!
% 時間の流れは途切れないように記述する!
\subsection{タイムスケジュール}
\begin{longtable}{p{3zw}p{39zw}}
	16:00 & \textbf{◎ 工科大到着} \\\\
	
	16:15 & \textbf{◎ 新入生見送り終了} \\
	& \ \ \textbullet \ \ 新入生全員の解散の確認後,スタッフの荷物,その他物品をK102に全員で運ぶ \\\\
	
	16:20 & \textbf{◎ 教室移動後} \\
	& \ \ \textbullet \ \ ゴミを指定の場所へ捨てに行く\\
	& \ \ \textbullet \ \ 分別ができていないものは分別する\\
	& \ \ \textbullet \ \ 各研究室,個人,教務部から借りた物品が揃っていることを確認し,順次返却を行う(教務部への返却は多田,和田が行う)\\
	& \ \ \textbullet \ \ 返却先が不在等で返却できない場合,一時的に清水研究室にて荷物を保管しておき,後日代表陣が物品の返却を行う\\\\
	
	16:40 & \textbf{◎ 反省会} \\\\
	
	17:00 & \textbf{◎ 解散!!} \\  
\end{longtable}

\subsection{人員配置}
%例:教務部への物品返却:多田,和田

% イベントに必要な物品と個数を記入する
% 記入例 ・マジックペン 10本
\subsection{必要物品}

% 注意事項やスタッフに周知しておくべきことがあれば記入する
\subsection{備考}

\end{document}