%\include{begin}

\section{イベント中裏方動き}

\subsection{日時・場所}
\begin{tabular}{p{2zw}rp{38zw}}
  日時 & : & 宿泊棟2階
\end{tabular}

\subsection{目的}
2日目のイベントの裏で,円滑に退所出来るように宿泊棟の整頓を行う

\subsection{タイムスケジュール}
% 時刻は必ず4桁(00:00)で書くこと!!!
\begin{longtable}{p{3zw}p{39zw}}
 

  08:50 & \textbf{◎ シーツ運び} \\
        & \ \ \textbullet \ \ 全員が体育館に移動したら,シーツをシーツ置き場に返却する \\\\
        
  09:00 & \textbf{◎ 退所点検} \\
        & \ \ \textbullet \ \ M2は事務室に職員を呼びに行く \\
        & \ \ \textbullet \ \ 退所点検終了次第,報告slackに,体育館に向かう旨を伝え向かう \\
        & \ \ \textbullet \ \ 終了次第イベント班15,16 に参加する\\
\end{longtable}


\subsection{人員配置}
\begin{itemize}
\item 裏方:M2
\end{itemize}

\subsection{退所点検}
\begin{itemize}
\item シーツ・枕カバをきれいにたんでおく
\item 寝具・備品(ハンガー等)は元通りにする
\item 窓は閉めて,カーテンは開けて留めておく
\item 室内・廊下の掃除をし,ゴミ箱のゴミは研修生入り口のゴミ置き場に分別して処理する
\item 電気・エアコンの消し忘れをしない
\end{itemize}


\subsection{備考}
\begin{itemize}
  \item 作業終了後は体育館に入り,後ろで待機する
  \item 忘れ物を発見した場合は,宮尾に報告し,体育館まで運ぶ
  \item 忘れ物があった場合,司会はイベントの休憩時間等に忘れ物があったことをアナウンスする
\end{itemize}

%\include{end}


