%\include{begin}

\section{イベント中裏方動き}

\subsection{日時・場所}
\begin{tabular}{p{2zw}rp{38zw}}
  日時 & : & 2019年4月6日(土) 09:00 $\sim$ 10:30\\
  場所 & : & くろしお棟
\end{tabular}

\subsection{目的}
2日目のイベントの裏で,円滑に退所出来るように宿泊棟の整頓を行う.


\subsection{タイムスケジュール}
% 時刻は必ず4桁(00:00)で書くこと!!!
\begin{longtable}{p{3zw}p{39zw}}
  09:00 & \textbf{◎ 退所点検} \\
        & \ \ \textbullet \ \ 各自,担当棟の掃除をする\\
        & \ \ \textbullet \ \ 前日の本音トークのゴミを福永車に運ぶ\\
        & \ \ \textbullet \ \ 担当棟の掃除が終わった人は他の棟の手伝いにまわる\\
        & \ \ \textbullet \ \ 全棟の掃除が終了次第,松本は事務室に職員を呼びに行く\\
        & \ \ \textbullet \ \ くろしお棟1から退所点検を行って頂く\\
        & \ \ \textbullet \ \ くろしお棟1,2,3,4いずれかで直しが多い場合は協力してベッドを直す\\\\
        & \ \ \textbullet \ \ 福永は全体を見て手伝いを行う\\
        & \textbf{◎ 宴のゴミの片づけ}\\
        & \ \ \textbullet \ \ 三浦,山口は宴のゴミの片づけを行う\\
        & \ \ \textbullet \ \ 終了次第,つどいの広間に移動し,イベントに参加する\\\\

  10:00 & \textbf{◎ シーツ運び} \\
        & \ \ \textbullet \ \ 全ての棟の退所点検が完了したら斎藤,松林,下出,松本でシーツを食堂横のシーツ置き場に返却する\\
        & \ \ \textbullet \ \ シーツ返却完了時に,松本は報告LINEに退所点検が終了したことを連絡する\\\\

  10:30 & \textbf{◎ 裏方終了} \\
        & \ \ \textbullet \ \ 福永は報告LINEに,つどいの広間に向かう旨を伝え全員で向かう\\
        & \ \ \textbullet \ \ 終了次第イベントを陰から見守る\\
\end{longtable}


\subsection{人員配置}
\begin{itemize}
  \item くろしお棟1:下出
  \item くろしお棟2:松林
  \item くろしお棟3:斎藤
  \item くろしお棟4:松本
  \item 宴のゴミの片付け:三浦(車),山口
\item 裏方統括:福永
\end{itemize}


\subsection{備考}
\begin{itemize}
  \item 作業終了後はつどいの広間に入り,後ろで待機する
  \item 忘れ物を発見した場合は,福永が野田に報告し,つどいの広間まで運ぶ
  \item 忘れ物があった場合,司会(横田,長通)はイベントの休憩時間等に忘れ物があったことをアナウンスする
\end{itemize}

%\include{end}


