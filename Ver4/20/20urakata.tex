%\include{begin}

\section{イベント中裏方動き}

\subsection{日時・場所}
\begin{tabular}{p{2zw}rp{38zw}}
  日時 & : & 宿泊棟2階
\end{tabular}

\subsection{目的}
2日目のイベントの裏で,円滑に退所出来るように宿泊棟の整頓を行う

\subsection{タイムスケジュール}
% 時刻は必ず4桁(00:00)で書くこと!!!
\begin{longtable}{p{3zw}p{39zw}}
  09:00 & \textbf{◎ 退所点検} \\
        & \ \ \textbullet \ \ 各自,宿泊部屋の掃除をする \\
        & \ \ \textbullet \ \ 全棟の掃除が終了次第,M2は事務室に職員を呼びに行く \\
        & \ \ \textbullet \ \ 終了次第,体育館に移動し,イベントに参加する \\\\

  10:00 & \textbf{◎ シーツ運び} \\
        & \ \ \textbullet \ \ 全ての棟の退所点検が完了したら,シーツを食堂横のシーツ置き場に返却する \\
        & \ \ \textbullet \ \ シーツ返却完了時に,報告slackに退所点検が終了したことを連絡する \\\\

  10:30 & \textbf{◎ 裏方終了} \\
        & \ \ \textbullet \ \ 報告slackに,体育館に向かう旨を伝え向かう\\
        & \ \ \textbullet \ \ 終了次第イベントを陰から見守る\\
\end{longtable}


\subsection{人員配置}
\begin{itemize}
\item 裏方:M2
\end{itemize}

\subsection{備考}
\begin{itemize}
  \item 作業終了後は体育館に入り,後ろで待機する
  \item 忘れ物を発見した場合は,宮尾に報告し,体育館まで運ぶ
  \item 忘れ物があった場合,司会はイベントの休憩時間等に忘れ物があったことをアナウンスする
\end{itemize}

%\include{end}


