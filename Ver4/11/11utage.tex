%\include{begin}

\section{夜の宴}
\subsection{日時・場所}
\begin{tabular}{p{2zw}rp{38zw}}
  日時 & : & 2019年4月5日(金) 19:00 $\sim$\\
  場所 & : & くじら公園
\end{tabular}

\subsection{目的}
日頃の感謝の気持ちを込めて先生方にお酒の場を提供し,学生と先生方の交流を深める.学生は先生方から教育に対する想いの丈や人生の先輩としてのお話を伺うことで,今後の学生生活および社会での生活に活かす.

\subsection{タイムスケジュール}
\begin{longtable}{p{3zw}p{39zw}}
  19:00 & \textbf{◎ 宴開始} \\
        & \ \  \textbullet \ \ 野外炊事が終わり次第,くじら公園にて宴を開始する \\
        & \ \  \textbullet \ \ 入浴を希望かどうか確認し,先生は後から参加する旨を伝え,青少年の家で待機してもらう \\
	    & \ \  \textbullet \ \ 修士生は準備ととも先生方にお供し、目的達成に務める \\\\
	    
	    
  21:30 & \textbf{◎ 宴終了} \\
        & \ \  \textbullet \ \ 青少年の家の本館は22:00に施錠されるので,早めに片付け,先生の誘導を始める \\
        & \ \  \textbullet \ \ 公園にゴミを残さないように片付けはしっかりする \\
  	    & \ \  \textbullet \ \ 状況により,青少年の家のスタッフ(小松)に連絡を取り,片付けを手伝いに来てもらう \\
  	    & \ \  \textbullet \ \ 入浴を希望する先生がいたら,大浴場が24:00まで使えることを伝える \\
  	    & \ \  \textbullet \ \ 3人のうち1人は大浴場の見張りを行い,2人は手の空いているスタッフとゴミをまとめる \\
\end{longtable}


\subsection{人員配置}
\begin{itemize}
\item 他の作業がない満20歳以上のスタッフ
\item 喫煙スタッフ:三浦,野田,以西
\item 迎えの車(必要なら):小島,西森,堀川,小松
\item 連絡係(青少年の家):小松
\end{itemize}


\subsection{必要物品}
\begin{itemize}
  \item 酒
  \item つまみ
  \item お冷
  \item 氷
  \item 取り皿
  \item ゴミ袋
  \item クーラーボックス
  \item キッチンペーパー
  \item 紙コップ(先生方が大量に消費するため去年より大量に)
\end{itemize}


\subsection{備考}
\begin{itemize}
	\item 翌日に車を運転するあるいは早朝に仕事が割り振られているスタッフは,作業に支障をきたさないようにする
	\item 飲み過ぎにより,先生方・スタッフ・新入生に迷惑をかけない(特にスタッフは飲みすぎて明日のイベントに支障をきたす恐れがあるため飲み過ぎ厳禁)
	\item 女子スタッフは21:00をめどに,自分の部屋へ戻る(22:00には撤収)
	\item 棟内は禁煙であるため,喫煙は所定の場所へ移動する(喫煙スタッフ適宜お供に付く)
	\item ゴミを分別する
	\item 酒類のゴミ(缶や注がれた紙コップ)は,一度水洗いしてゴミ袋に入れる
	\item スタッフは率先して先生方にお酒をお注ぎする
	\item 22:00に玄関のかぎが施錠されるので,それ以前に青少年の家へ戻る
	\item 現地スタッフだけで片付けが困難な場合は,小松に連絡をする
	\item 小松は連絡が来たら手の空いているスタッフに手伝いに行くよう伝える
	%\item 施設全体として原則禁酒であるため,お酒をくろしお棟4以外から持ち出さない(持ち出そうとした場合は止める)
	%\item 新入生をくろしお棟4へ侵入させない
    %\item 体調の悪い人は部屋で待機する(くろしお棟4の1階にいると飲まされる可能性があるため)
    %\item 先生方がお酒を用意してくださるのでそこまでお酒を購入しなくても良い?
\end{itemize}

%\include{end}
