% --*- coding:utf-8-unix mode:latex -*--
%\include{Begin}
%%%%%%%%%%%%%%%%%%%%%%%%%%%%%%%%%%%%%%%%%%%%%%%%%%%%%%%%%%%%%%%%%%%%%%%%%%%%%%%

\section{荷物の搬入}

\subsection{日時・場所}

\begin{tabular}{p{2zw}rp{38zw}}
  日時 & : & 2019年4月5日(金) 7:50$\sim$ 8:30\\
  場所 & : & A109,門田研究室,妻鳥研究室,吉田研究室
\end{tabular}

\subsection{目的}
迅速に荷物を運ぶ!!!!


\subsection{タイムスケジュール}
% 時刻は必ず4桁(00:00)で書くこと!!!
\begin{longtable}{p{3zw}p{39zw}}
   7:50 & \textbf{◎ 搬入開始} \\
        & \ \  \textbullet \ \ 小島,小松は車をA棟西側に持ってくる \\
        & \ \  \textbullet \ \ 横田,野田,以西は浜村研究室からA棟西側に大学から持ち出す荷物を搬入する \\
        & \ \  \textbullet \ \ 小島,小松は車を停車後,門田,妻鳥,吉田研究室にワゴン車に載せる荷物を取りに行き,A109に搬入する\\\\

  8:20 & \textbf{◎ 搬入終了} \\
        & \ \  \textbullet \ \ 再度,本企画書と搬入した物品を確認する \\
        & \ \  \textbullet \ \ 物品がない場合は先遣隊が購入するためリストアップする \\\\

 8:30 & \textbf{◎ 先遣隊出発} \\
        & \ \  \textbullet \ \ 横田が報告slackに連絡した後,出発する \\
\end{longtable}


\subsection{人員配置(人数により調整,運転者含む)}
\begin{itemize}
\item 先遣隊1:小島(車),野田,宮尾
\item 先遣隊2:小松(車),横田,以西

\end{itemize}

\subsection{備考}
\begin{itemize}
\item 前日までに荷物を門田,妻鳥,吉田研究室へ運んでおく
\item 研究室へ搬入の際にも物品を確認しておく
\item バス司会(塩谷,中島,丸田,高橋,藤田,北村)はスタッフ集合部屋で酔い止めと水を受け取る
\end{itemize}


%%%%%%%%%%%%%%%%%%%%%%%%%%%%%%%%%%%%%%%%%%%%%%%%%%%%%%%%%%%%%%%%%%%%%%%%%%%%%%%
%\include{End}
