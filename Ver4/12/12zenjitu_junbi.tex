% 必要な項目ができた場合は適宜サブセクションを追加してください

%\include{begin}

% イベント名を記入する
\section{前日準備}


% 日時と場所を記入する
% 時刻は4桁で記入すること!
\subsection{日時・場所}
\begin{tabular}{p{2zw}rp{38zw}}
  日時 & : & 2019年4月5日(金) 19:00 $\sim$ 20:00\\
  場所 & : & 体育館
\end{tabular}


% 目的を記入する
\subsection{目的}
翌日のイベント開始前に行う準備作業を減らす.

\subsection{内容}
野外炊事が終わり担当グループの新入生を各部屋まで誘導後,ロビーに集合する \\
第4研修室から必要な機材を体育館に運び出し,???は機材(パソコン,カメラ等)の動作確認をする \\
机,椅子を体育館に配置する \\
イベントで使う備品は第4研修室にまとめて置いておく \\
プレゼンターは体育館で生活紹介のリハーサルを行う \\
タイムスケジュールに記載しているように,20:00まで作業可能とする \\
作業が早く終了したらその時点で終了とする \\
現場は宮尾の指揮で動く \\

% イベントのタイムスケジュールを記入する
% 時刻は必ず4桁(00:00)で記入すること!
% 時間の流れは途切れないように記述する!
\subsection{タイムスケジュール}
\begin{longtable}{p{3zw}p{39zw}}
  19:00 & \textbf{◎ スタッフ集合・作業開始} \\
        & \ \ \textbullet \ \ 宮尾が報告slackに連絡する\\
        & \ \  \textbullet \ \ ロビーに集合次第,作業開始 \\\\
  20:00 & \textbf{◎ 作業終了} \\
        & \ \  \textbullet \ \ 最長作業終了時間\\
\end{longtable}


% イベントに必要な役割と人数を記入する
% 担当者は決定次第追記する
% 記入例 ・司会者 2人(名前1、名前2)
\subsection{人員配置}
\begin{itemize}
\item 鍵係:
\item イベント統括:宮尾
\item 機器操作:
\item 総合司会:
\item 生活紹介リハーサル:
\item 各イベントプレゼンター:
\end{itemize}


% イベントを実施するときに新入生や先生、スタッフがどこに配置するかを記入する
% 図があるとわかりやすい
%\subsection{全体配置}


% イベントに必要な物品と個数を記入する
% 記入例 ・マジックペン 10本
\subsection{必要物品}
\begin{itemize}
\item 机:3つ
\item 椅子:1つ
\item プロジェクタ
\item PC(高橋(練)のPCを使用)
\item カメラ(iPhoneを使用)
\item 延長ケーブル
\item スケッチブック:7冊
\item イベント班を示したプラカード:14個
\item 景品
%\item 赤白帽:7つ
%\item ヒントを書く小さい紙:84枚
%\item ヒントをまとめる大きい紙:28枚
\item マジック:14本
\end{itemize}


% 注意事項やスタッフに周知しておくべきことがあれば記入する
\subsection{備考}
\begin{itemize}
\item 机,椅子は体育館にあるものを使用する
\item マイクの電池を確認しておく
%\item 昨年,つどいの広間の鍵を借り忘れて作業ができなかったので,鍵係の岡本は鍵を借りることを忘れない!
\item 宮尾はすぐ連絡が取れるようにスマホを常備しておく
\end{itemize}

%\include{end}

