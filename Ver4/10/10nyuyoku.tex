% 必要な項目ができた場合は適宜サブセクションを追加してください
%\include{begin}
% イベント名を記入する
\section{入浴}


% 日時と場所を記入する
% 時刻は4桁で記入すること!
\subsection{日時・場所}
\begin{tabular}{p{2zw}rp{38zw}}
  日時 & : & 2019年4月5日(金) 18:20 $\sim$ 21:00\\
  場所 & : & 大浴場,小浴場,宿泊部屋
\end{tabular}


% 目的を記入する
\subsection{目的}
新入生・先生の入浴がスムーズに行えるようにする.

% イベントの概要やルールを記入する
\subsection{内容}
野外炊事から戻り,ベッドメイキング,新入生全員の入浴が終わるまでの流れを説明する.スタッフは適宜時間の空いた人から入浴を済ませる(女性スタッフ(小浴場) 19:00  $\sim$ 20:15,
\ 男性スタッフ(大浴場) 19:30 $\sim$ 21:00
).入浴できなかったスタッフは指導者棟で入浴する.また,B3以上のスタッフは,入浴前後の空いてる時間に,先生方の宴のお相手をする.


% イベントのタイムスケジュールを記入する
% 時刻は必ず4桁(00:00)で記入すること!
% 時間の流れは途切れないように記述する!
\subsection{タイムスケジュール}
\begin{longtable}{p{3zw}p{39zw}}
  18:20 & \textbf{◎ 女性教職員の入浴開始}\\
        & \ \ \textbullet \ \ 女性教員誘導係(江川)は女性教職員を小浴場に誘導する\\
        & \ \ \textbullet \ \ 女性教職員誘導係は女性教員に19時までに出るように伝える \\
        & \ \ \textbullet \ \ 誘導後,江川は小浴場の見張りをする\\
        & \ \ \textbullet \ \ 女性教職員全員の入浴が終了後,女性教員誘導係(江川)は入浴終了の旨を報告LINEに連絡し,くろしお棟1に移動する\\
        & \ \ \textbullet \ \ 女性スタッフは、女性教職員の入浴終了のLINEが来次第、随時入浴を開始する\\\\


  19:00 & \textbf{◎ ベッドメイキング指導開始(随時行う)} \\
        & \ \ \textbullet \ \ くろしお棟1,2,3のスタッフ(小松,角原,平松,齋藤,宮尾,明神)は各棟1Fにて新入生のベッドメイキング指導を行う\\
        & \ \ \textbullet \ \ 修士生は野外炊事終了後,教職員のベッドメイキングを行う\\
        & \ \ \textbullet \ \ 教職員のベッドメイキングの人員が足りない場合は手の空いているB4が手伝う\\
        & \ \ \textbullet \ \ 各くろしお棟のスタッフは新入生に19:15(くろしお棟2は20:15)まで各くろしお棟から出ないことを注意する \\
        & \ \ \textbullet \ \ 新入生自身が持参したドライヤーは使用しないように注意しておく\\
        & \ \ \textbullet \ \ 大浴場または小浴場で入浴できない新入生がいた場合は,各くろしお棟スタッフが連絡する\\\\

  19:30 & \textbf{◎ 1回目入浴開始} \\
        & \ \ \textbullet \ \ 各くろしお棟スタッフは\ref{sec:bath}節に示した入浴開始時間の10分前には入浴準備をするよう新入生を促し,浴場へ誘導して入浴を開始する\\
        & \ \ \textbullet \ \ くろしお棟3は宮尾,明神が,くろしお棟1は小松,角原が誘導する\\
        & \ \ \textbullet \ \ 宮尾は新入生と一緒に入浴開始し,明神は大浴場にて見張りをする\\
        & \ \ \textbullet \ \ 小松は新入生と一緒に入浴開始し,角原は小浴場にて見張りをする \\
        & \ \ \textbullet \ \ もし大浴場で入浴できない新入生がいた場合は,宮尾は入浴せず,指導者棟・慎太郎の浴場に誘導して見張りをし,明神は連絡した上で大浴場の見張りをする \\
        & \ \ \textbullet \ \ くろしお棟2の新入生はベッドメイキングが終わったら1Fで談話する(強制はしない) \\
        & \ \ \textbullet \ \ もし小浴場で入浴できない新入生がいた場合は,小松は入浴せず,指導者棟・龍馬の浴場に誘導して見張りをし,角原は連絡した上で小浴場の見張りをする \\\\

        & \textbf{◎ 1回目入浴終了} \\
        & \ \ \textbullet \ \ 宮尾,小松はそれぞれ浴場の整理整頓と忘れ物の点検を行う\\
        & \ \ \textbullet \ \ 明神はくろしお棟2に新入生を呼びに行き,入浴終了の旨を報告LINEに連絡する\\
        & \ \ \textbullet \ \ 平松は指導者棟・慎太郎の鍵を明神から受け取る\\
        & \ \ \textbullet \ \ 角原はくろしお棟4に先生を呼びに行き,入浴終了の旨を報告LINEに連絡する\\
        & \ \ \textbullet \ \ 角原は指導者棟・龍馬の鍵を鍵係(岡本)に返却する\\
        & \ \ \textbullet \ \ 新入生が各くろしお棟に戻り次第,ベッドメイキングをするよう促す \\
        & \ \ \textbullet \ \ 翌日の荷物移動を速やかにするため,就寝前に荷造りをすませておくよう伝える\\
        & \ \ \textbullet \ \ 役割を終えて時間が空き次第,小松,角原は指導者棟・龍馬で,宮尾,明神は大浴場でくろしお棟2と一緒に入浴する(時間がなかった場合は指導者棟・慎太郎で入浴する)\\\\

  20:15 & \textbf{◎ 2回目入浴開始} \\
        & \ \ \textbullet \ \ くろしお棟1,3はベッドメイキングが終わっていない人がいれば指導する\\
        & \ \ \textbullet \ \ くろしお棟2は明神が呼びに来次第平松,齋藤が誘導,くろしお棟4は角原が呼びに来次第松本が誘導する \\
        & \ \ \textbullet \ \ 平松は新入生と一緒に入浴開始し,齋藤は大浴場にて見張りをする\\
        & \ \ \textbullet \ \ 松本は小浴場にて見張りをする \\
        & \ \ \textbullet \ \ もし大浴場で入浴できない新入生がいた場合は,平松は入浴せず,指導者棟・慎太郎の浴場に誘導して見張りをし,齋藤は報告LINEに連絡した上で大浴場の見張りをする \\
        & \ \ \textbullet \ \ くろしお棟4にて,入浴より宴を優先される先生方には入浴時間が21:00までであることを伝える\\
        & \ \ \textbullet \ \ この時間までに小浴場で入浴できそうにない女性スタッフは,指導者棟・龍馬で入浴する\\
        & \ \ \textbullet \ \ 女性スタッフ全員入浴を終え次第,指導者棟・龍馬を入浴時間内に入れない男性スタッフに解放する\\\\


        & \textbf{◎ 2回目入浴終了} \\
        & \ \ \textbullet \ \ 平松,松本はそれぞれ浴場の整理整頓と忘れ物の点検を行う\\
        & \ \ \textbullet \ \ 齋藤は報告LINEに入浴終了の旨を連絡し,新入生がくろしお棟に戻り次第,ベッドメイキングをするよう促す\\
  & \ \ \textbullet \ \ 平松は指導者棟・慎太郎の鍵を鍵係岡本に返却する\\
        & \ \ \textbullet \ \ 役割を終えて時間が空き次第,齋藤,松本は指導者棟で入浴する \\
        & \ \ \textbullet \ \ 翌日の荷物移動を速やかにするため,就寝前に荷造りをすませておくよう伝える \\
        & \ \ \textbullet \ \ 役割を終え,時間が空き次第指導者棟で入浴する\\\\

  21:00 & \textbf{◎ 入浴終了} \\
        & \ \ \textbullet \ \ この時間までに大浴場で入浴できそうにない男性スタッフは,指導者棟・慎太郎で入浴する(指導者棟・龍馬が解放されていたらそちらも利用する)\\\\
\end{longtable}



% イベントに必要な役割と人数を記入する
% 担当者は決定次第追記する
% 記入例 ・司会者 2人(名前1、名前2)
\subsection{人員配置}
○入浴誘導
\begin{itemize}
 \item くろしお棟1:小松,角原
 \item くろしお棟2:平松,齋藤
 \item くろしお棟3:明神,宮尾
 \item くろしお棟4:松本
 \item 女性教職員誘導:江川
\end{itemize}
○ベットメイキング指導
\begin{itemize}
\item くろしお棟1:小松,角原
 \item くろしお棟2:平松,齋藤
 \item くろしお棟3:明神,宮尾
\item その他スタッフ:先生のベットメイキング
\end{itemize}
% イベントに必要な物品と個数を記入する
% 記入例 ・マジックペン 10本
\subsection{必要物品}
\begin{itemize}
\item ドライヤー:5つ
%\item シャンプー
%\item ボディーソープ
\end{itemize}

\subsection{各棟入浴開始時間}
\label{sec:bath}
\begin{table}[h]
\begin{tabular}{|c|c|c|l|}
\hline
{棟名}&{浴場}&{入浴開始時間}&\multicolumn{1}{c|}{対象} \\ \hline\hline
           & 小浴場 & 18:20 & 女性教職員 \\ \hline
くろしお棟1 & 小浴場 & 19:30 & 新入生(女性),女性スタッフ \\ \hline
くろしお棟2 & 大浴場 & 20:15 & 新入生(男性),男性スタッフ \\ \hline
くろしお棟3 & 大浴場 & 19:30 & 新入生(男性) \\ \hline
くろしお棟4 & 小浴場 & 20:15 & 男性教職員 \\ \hline
\end{tabular}
%%\label{tab:bath}
%%\caption{各棟入浴開始時間}
\end{table}
% 注意事項やスタッフに周知しておくべきことがあれば記入する
\subsection{備考}
\begin{itemize}
\item 小松(くろしお棟1担当スタッフ),宮尾(くろしお棟3担当スタッフ)は野外炊事から荷物を取りに行く時,第一集会室にいる岡本から指導者棟(龍馬・慎太郎)の鍵をもらう
\item 基本的に,小松,宮尾が指導者棟の鍵を岡本に返却するまで,スタッフは指導者棟を使用しない
\item その他,指導者棟の鍵が必要な人は鍵係(岡本)に連絡して鍵をもらう(指導者棟を利用する場合は,適宜報告LINEに連絡する)
\item 時間や手間の都合上鍵を又貸しする場合は,その旨を報告LINEに報告して誰が鍵をもっているのか明確にしておく
\item 鍵係は常に連絡がとれる状況にしておく
\end{itemize}

% \subsection{連絡事項}
% \begin{table}[h]
% \begin{tabular}{|c|c|c|c|}
% \hline
% {報告者}&{内容}&{タイミング}&{備考}\\ \hline\hline
% Aさん & 指導者棟・慎太郎の浴場使用 & 1回目入浴開始時 & 大浴場で入浴できない新入生がいた場合のみ\\ \hline
% Bさん & 指導者棟・龍馬の浴場使用 & 1回目入浴開始時 & 小浴場で入浴できない新入生がいた場合のみ\\ \hline
% Cさん & くろしお棟1の入浴終了 & 1回目入浴終了時 & \\ \hline
% Dさん & くろしお棟3の入浴終了 & 1回目入浴終了時 & \\ \hline
% Fさん & 指導者棟・慎太郎の浴場使用 & 2回目入浴開始時 & 大浴場で入浴できない新入生がいた場合のみ\\ \hline
% Gさん & くろしお棟2の入浴終了 & 2回目入浴終了時 & \\ \hline
% \end{tabular}
% \label{tab:bath}
% \caption{各棟入浴開始時間}
%\end{table}

%\include{end}
